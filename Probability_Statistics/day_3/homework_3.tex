\documentclass[12pt]{article}

\usepackage{mathrsfs}
\usepackage{epsfig}
\usepackage{graphicx}
\usepackage{color}
\usepackage{amsmath}
\usepackage{amsfonts}
\usepackage{amssymb}
\usepackage{amsthm}
\usepackage{amscd}
\usepackage{verbatim}
\usepackage{fullpage}
\usepackage{indentfirst}

%\setcounter{MaxMatrixCols}{20}
\theoremstyle{plain}
\newtheorem{theorem}{Theorem}[section]
\newtheorem{proposition}[theorem]{Proposition}
\newtheorem{lemma}[theorem]{Lemma}
\newtheorem{corollary}[theorem]{Corollary}
\newtheorem{conjecture}[theorem]{Conjecture}
\theoremstyle{definition}
\newtheorem{definition}[theorem]{Definition}
\newtheorem{notation}[theorem]{Notation}
\newtheorem{remark}[theorem]{Remark}
\newtheorem{example}[theorem]{Example}
\numberwithin{equation}{theorem}

\newcommand{\black}{\hfill{\ensuremath{\blacksquare}}}
\DeclareMathAlphabet{\mathpzc}{OT1}{pzc}{m}{it}

\begin{document}
\begin{titlepage}
	\centering
	\vspace{4cm}
	{\scshape\Large DSE 210\par}
	\vspace{1.5cm}
	{\huge\bfseries Homework 3\par}
	\vspace{2cm}
	{\Large\itshape Kevin Kannappan\par}

% Bottom of the page
	{\large \today\par}
\end{titlepage}


\section{Generative models 2}
\subsection{Worksheet 6}
\begin{enumerate}
\item
	\begin{enumerate}
	\item Uncorrelated
	\item Positively Correlated
	\item Negatively Correlated
	\end{enumerate}
\addtocounter{enumi}{1}
\item
	\begin{enumerate}
	\item Unique Bivariate Gaussian, parameterized bymean: $\mu = \binom{2}{2}$ and Covariance Matrix:
	\[
Cov(x,y)=
  \begin{bmatrix}
    1 & -0.25 \\
    -0.25 & 0.25
  \end{bmatrix}
\]\\
	Because each standard deviation squared provides the variance, we are able to get the diagonals of the matrix. Calculating the covariance comes from the correlation formula: $corr(x,y) = \frac{cov(x,y)}{std(x)*std(y)}$.
	\item Unique Bivariate Gaussian, parameterized bymean: $\mu = \binom{1}{1}$ and Covariance Matrix:
	\[
Cov(x,y)=
  \begin{bmatrix}
    1 & 1 \\
    1 & 1
  \end{bmatrix}
\]\\
	Because each standard deviation squared provides the variance (1:1), we are able to get the diagonals of the matrix. Since $y=x$, intuitively we know that their covariance must equal 1.
	\end{enumerate}
\bigskip
Please see attached Jupyter notebook for questions 4 and 5.
\end{enumerate}
\bigskip

\section{Linear Algebra Primer}
\subsection{Worksheet 7}
\begin{enumerate}
\item $||x|| = \sqrt{1^{2}+2^{2}+3^{2}} = \sqrt{14}$. Hence the unit vector in the same direction as x is: $\vec{x_{u}} = (\frac{1}{\sqrt{14}},\frac{2}{\sqrt{14}},\frac{3}{\sqrt{14}})$.
\item Dot product must equal 0 for orthogonality, hence need to see a relation like: $\binom{1}{1}\cdot \binom{-1}{1} = -1+1 = 0$. Taking the unit vectors, we get: $(\frac{-1}{\sqrt{2}},\frac{1}{\sqrt{2}})$ and $(\frac{1}{\sqrt{2}},\frac{-1}{\sqrt{2}})$.
\item $x\cdot x = 25$, hence $||x||^{2}=x \cdot x$ and thus 5 is the magnitude. In 2 dimensions, we know that this is a circle with radius 5 and in 3 dimensions this would be a sphere with radius 5. Because we have $d$ dimensions, this would be classified as a hypersphere with radius 5.
\item 
\begin{align}
    w &= \begin{bmatrix}
           2 \\
           -1 \\
           6
         \end{bmatrix}
 \end{align}
\item Following matrix multiplication properties, we know $A = 10 \times 30$ and $B = 30 \times 20$.
\item
	\begin{enumerate}
	\item $n \times d$.
	\item $n \times n$.
	\item $x^{i} \cdot x^{j}$.
	\end{enumerate}
\addtocounter{enumi}{1}
\item $x^{T}x = ||x||^{2}$. Hence, we have $\sqrt{1^{2}+3^{2}+5^{2}}^{2} = 35$. $xx^{T}$ is a matrix multiplied by its transpose:\\
	\[
xx^{T}=
  \begin{bmatrix}
    1 & 3 & 5 \\
    3 & 9 & 15 \\
    5 & 15 & 25
  \end{bmatrix}
\]
\addtocounter{enumi}{1}
\item Writing a symmetric matrix, we have:\\
	\[
M=
  \begin{bmatrix}
    3 & 1 & -2 \\
    1 & 0 & 0 \\
    -2 & 0 & 6
  \end{bmatrix}
\]
\item a) and c). a) because of multiplicative properties and c) because of additive properties. Not d) because of negative values within the matrix.
\addtocounter{enumi}{1}
\item
	\begin{enumerate}
	\item $UU^{T}$ yields the $d \times d$ identity matrix, $I^{d}$.
	\item Following part a), we know from matrix properties that any matrix $M$, multiplied by its inverse $M^{-1}$, yields the identity matrix. Hence, we can conclude that $U^{-1} = U^{T}$.
	\end{enumerate}
\item $z = 6$, since the discriminant must = 0. Leaves us with $z-6=0$ and 6.
\end{enumerate}
\bigskip

\section{Classification with Generative Models 3}
\subsection{Worksheet 8}
\bigskip
Please see attached Jupyter notebook for digit classification.


\end{document}