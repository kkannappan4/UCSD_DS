\documentclass[12pt]{article}

\usepackage{mathrsfs}
\usepackage{epsfig}
\usepackage{graphicx}
\usepackage{color}
\usepackage{amsmath}
\usepackage{amsfonts}
\usepackage{amssymb}
\usepackage{amsthm}
\usepackage{amscd}
\usepackage{verbatim}
\usepackage{fullpage}
\usepackage{indentfirst}

%\setcounter{MaxMatrixCols}{20}
\theoremstyle{plain}
\newtheorem{theorem}{Theorem}[section]
\newtheorem{proposition}[theorem]{Proposition}
\newtheorem{lemma}[theorem]{Lemma}
\newtheorem{corollary}[theorem]{Corollary}
\newtheorem{conjecture}[theorem]{Conjecture}
\theoremstyle{definition}
\newtheorem{definition}[theorem]{Definition}
\newtheorem{notation}[theorem]{Notation}
\newtheorem{remark}[theorem]{Remark}
\newtheorem{example}[theorem]{Example}
\numberwithin{equation}{theorem}

\newcommand{\black}{\hfill{\ensuremath{\blacksquare}}}
\DeclareMathAlphabet{\mathpzc}{OT1}{pzc}{m}{it}

\begin{document}
\begin{titlepage}
	\centering
	\vspace{4cm}
	{\scshape\Large DSE 210\par}
	\vspace{1.5cm}
	{\huge\bfseries Homework 2\par}
	\vspace{2cm}
	{\Large\itshape Kevin Kannappan\par}

% Bottom of the page
	{\large \today\par}
\end{titlepage}


\section{Random Variable, Expectation, and Variance}
\subsection{Worksheet 4}
\begin{enumerate}
\item Let $X = min(X_{1},X_{2})$ where $X_{1}$ and $X_{2}$ are the outcomes of respective dice rolls.\\
$\Omega = \{1,2,3,4,5,6\}$. Beginning with $6 = min(X_{1},X_{2})$, we know that there is one possibility, $\{6,6\}$.\\
Hence, we can define the following distribution of $X$, decreasing from 6:\\
\\
\begin{tabular}{ |p{2.5cm}|p{2cm}|p{2cm}|p{2cm}|p{2cm}|p{2cm}|p{2cm}|  }
 \hline
 \multicolumn{7}{|c|}{Distribution of $X$} \\
 \hline
 $min(X_{1},X_{2})$&1&2&3&4&5&6 \\
 \hline
 $Pr(X =x)$&$\frac{11}{36}$&$\frac{1}{4}$&$\frac{7}{36}$&$\frac{5}{36}$&$\frac{1}{12}$& $\frac{1}{36}$ \\
 \hline
\end{tabular}
\item $6$. Since, $6 \times \frac{1}{6} = 1$. 
\addtocounter{enumi}{1}
\item
	\begin{enumerate}
	\item $\binom{n}{1}*\frac{1}{10}*\frac{9}{10}^{n-1}$ as there are $\binom{n}{1}$ ways of selecting an individual to get off at a particular floor with equal probability $\frac{1}{10}$, with the remaining $n-1$ individuals selecting from $\frac{9}{10}$ floors.
	\item $10*\binom{n}{1}*\frac{1}{10}*\frac{9}{10}^{n-1}$. Since each floor has an equal probability, it is like a coin flip, hence it is $n*p$, for which $n=10$.
	\end{enumerate}
\addtocounter{enumi}{1}
\item Let $X$ represent the number of people who end up in their own bed. Hence, \\
$X = x_{1}+x_{2}+\dots+x_{n}$ where
\begin{align}
x_{i} & = \begin{cases}
                1, & $if person ends in the 1st bed$\\
                 0, & $otherwise.$
                    \end{cases}
\end{align}
Now, we have $Pr(x_{i}=1)=\frac{1}{n}$, which, by the linearity of expectation yields $E(x_{i})=\frac{1}{n}*n = 1$. The expected number of students who end up in their own bed is $1$.
\item
	\begin{enumerate}
	\item $Pr(X) = \frac{1}{n}$ and $Pr(Y) = \frac{1}{n}$. If $X=Y$, then $Pr(X\cap Y)=0$ because $X \neq Y$ by permutation principles. Hence, $X$ and $Y$ are dependent.
	\addtocounter{enumii}{1}
	\item $Pr(X\cap Y) = \frac{1}{52}$, since there is only one card that is both a 9 and a heart. If the result is independent, then:\\
	$Pr(X\cap Y) = Pr(X)*Pr(Y)$ Computing the probabilities we have:\\
	$Pr(X) = \frac{1}{13}$, $Pr(Y) = \frac{1}{4}$. Lastly, \\
	$Pr(X\cap Y) = \frac{1}{13} * \frac{1}{4} = \frac{1}{52}$. $X$ and $Y$ are independent.
	\end{enumerate}
\item
	\begin{enumerate}
	\item $E(Z) = 1\times \frac{1}{8} + 2\times \frac{1}{8} + 3\times \frac{1}{8} + 4\times \frac{1}{8} + 5\times \frac{1}{4} + 6\times \frac{1}{4}$, or \\
$E(Z) = \frac{10}{8} +  \frac{10}{8} +  \frac{12}{8} = 4$. $E(Z)=4$.\\
$Var(Z) = E(Z^{2})-\mu^{2}$. Computing it out we have:\\
$Var(Z) = 1\times \frac{1}{8} + 4\times \frac{1}{8} + 9\times \frac{1}{8} + 16\times \frac{1}{8} + 25\times \frac{1}{4} + 36\times \frac{1}{4} - 4^{2}$\\
$Var(Z) = 19-16 = 3$. $Var(Z) = 3$.
	\item By expected value and variance rules, I expect $E(X) = 40$ and $Var(X) = 30$. Written out we see:\\
	$E(X) = E(X_{1})+E(X_{2}) + \dots + E(X_{10})$. Or, $E(X) = 10*4$. For variance of X, we have:\\
	$Var(X) = Var(X_{1})+Var(X_{2}) + \dots + Var(X_{10})$. Or, $Var(X) = 10*3$.
	\item Because we are taking the average of all the rolls, $A$, we know that it is the sum of the rolls, divided by the number of times the dice is rolled, $n$. $E(A) = 4$ because it is the definition of expectation. For variance, we know by the rules of variance that $var(aX+b)=a^{2}var(X)$. Leveraging this property, we can conclude $Var(A) = \frac{3}{n}$.
	\end{enumerate}
\addtocounter{enumi}{3}
\item There is a way to solve this problem using an infinite sum, considering that there is always a possibility of starting over (next coin mismatches the previous). However, by indicating that there is an initial dead toss, toss $+1$, then there is an expected number of tosses $2$ to generate that particular outcome. Hence, the expected number of tosses to yield a consecutive outcome is $1+2=3$.
\end{enumerate}
\bigskip

\section{Classification with generative models 1}
\subsection{Worksheet 5}
\begin{enumerate}
\item 
	\begin{enumerate}
	\item As the probability of talking a little, $Pr($talks a little$) = \frac{1}{6}$, is the same for both happy and sad classes, we can reason that his most likely mood is happy, since it has a higher fraction of the data with that label.
	\item $\frac{1}{4}$, since it is the probability of not being happy (sad).
	\end{enumerate}
\item \begin{align}
h^{*}(X ,Y) & = \begin{cases}
                Y= 1, & $for $X \in [-1,0)\\
                Y = 3, & $for$ X \in [0,1].
                    \end{cases}
             \end{align}
\end{enumerate}


\end{document}