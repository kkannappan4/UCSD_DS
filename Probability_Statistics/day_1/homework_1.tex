\documentclass[12pt]{article}

\usepackage{mathrsfs}
\usepackage{epsfig}
\usepackage{graphicx}
\usepackage{color}
\usepackage{amsmath}
\usepackage{amsfonts}
\usepackage{amssymb}
\usepackage{amsthm}
\usepackage{amscd}
\usepackage{verbatim}
\usepackage{fullpage}
\usepackage{indentfirst}

%\setcounter{MaxMatrixCols}{20}
\theoremstyle{plain}
\newtheorem{theorem}{Theorem}[section]
\newtheorem{proposition}[theorem]{Proposition}
\newtheorem{lemma}[theorem]{Lemma}
\newtheorem{corollary}[theorem]{Corollary}
\newtheorem{conjecture}[theorem]{Conjecture}
\theoremstyle{definition}
\newtheorem{definition}[theorem]{Definition}
\newtheorem{notation}[theorem]{Notation}
\newtheorem{remark}[theorem]{Remark}
\newtheorem{example}[theorem]{Example}
\numberwithin{equation}{theorem}

\newcommand{\black}{\hfill{\ensuremath{\blacksquare}}}
\DeclareMathAlphabet{\mathpzc}{OT1}{pzc}{m}{it}

\begin{document}
\begin{titlepage}
	\centering
	\vspace{4cm}
	{\scshape\Large DSE 210\par}
	\vspace{1.5cm}
	{\huge\bfseries Homework 1\par}
	\vspace{2cm}
	{\Large\itshape Kevin Kannappan\par}

% Bottom of the page
	{\large \today\par}
\end{titlepage}


\section{Sets and Counting}
\subsection{Worksheet 1}
\begin{enumerate}
\item
	\begin{enumerate}
	\item A = $\{a,b,c,d,e\}$; $|A|$ = 5
	\item $A^{3} = A \times A \times A$
	\item $5^{3} = 125$
	\end{enumerate}
\item $2^{500}$
\item
	\begin{enumerate}
	\item 7, $|A| + |B|$
	\item 4, $A \subset B$
	\item Largest = 3, $A \subset B$;  Smallest = $\o$, no intersection
	\end{enumerate}
\item $4! = 4*3*2 = 24$, order $4$ separate animals.
\item $26^{5} = 11,881,376$, alphabet sequence length $5$.
\item Pick $3$ items out of $10$ possibilities: $\binom{10}{3} = \frac{10*9*8}{3*2} = 120$
\item Pick $5$ items out of $10$ possibilites, order $5$ separate ways. $\binom{10}{5}*5! = \frac{10!}{5!*5!} = 10*9*8*7*6=30,240$
\end{enumerate}
\bigskip

\section{Probability Spaces}
\subsection{Worksheet 2}
\begin{enumerate}
\addtocounter{enumi}{1}
\item
	\begin{enumerate}
	\item $\Omega = \{H,T\}^{200}$; $|\Omega| = 2^{200}$
	\end{enumerate}
\item
	\begin{enumerate}
	\item $A\cap B \cap C$
	\item $A\cup B \cup C$
	\item $A\cap B \cap C^{c}$, where $C^{c}$ is the complement of $C$
	\end{enumerate}
\item
	\begin{enumerate}
	\item $Pr(c) = 1-(Pr(a)+Pr(b)) = \frac{1}{6}$
	\item 8; $\big\{\{\},\{a\},\{b\},\{c\},\{ab\},\{ac\},\{bc\},\{abc\}\big\}$
	\item $Pr(\{\})=0$, $Pr(\{a\})=\frac{1}{2}$, $Pr(\{b\})=\frac{1}{3}$,  $Pr(\{c\})=\frac{1}{6}$,  $Pr(\{ab\})=\frac{5}{6}$,  $Pr(\{ac\})=\frac{2}{3}$,  $Pr(\{bc\})=\frac{1}{2}$,  $Pr(\{abc\})=1$
	\end{enumerate}
\item
	\begin{enumerate}
	\item $E_{1}=$ ``The first shot is heads.'' $Pr(E_{1}) = \frac{4}{8} = \frac{1}{2}$
	\item $E_{2}=$ ``All three tosses are the same result.''  $Pr(E_{2}) = \frac{2}{8} = \frac{1}{4}$
	\item $E_{3}=$ ``There is exactly one tails.'' $Pr(E_{3}) = \frac{3}{8} $
	\end{enumerate}
\item $Pr(A\cup B) = Pr(A) + Pr(B) - Pr(A\cap B)\\
Pr(A^{c}) = 1-Pr(A)\\
\frac{1}{3} = 1-Pr(A)$; $Pr(A) = \frac{2}{3}$. Substitute values:\\
$Pr(A\cup B) = \frac{2}{3} + \frac{1}{2} - \frac{1}{4} = \frac{11}{12}$
\item $|\Omega| = 6^{2}$, $\Omega = \{1,2,3,4,5,6\}$; $6 x\times \frac{1}{6^{2}} = \frac{6}{36} = \frac{1}{6}$ 
\addtocounter{enumi}{1}
\item $100 = x + 2x + 3x + 4x + 5x + 6x\\
100 = 21x$; $Pr(1) = 1/21)$. Hence, $Pr(even)$ = $Pr(2,4,6) = \frac{2}{21} + \frac{4}{21} + \frac{6}{21}\\
= \frac{12}{21}$
\addtocounter{enumi}{1}
\item Only one possibility. Because the five people have different heights, the formula for the individuals being arranged in increasing order of height is $1 \times \frac{1}{5!} = \frac{1}{120}$
\addtocounter{enumi}{2}
\item $Pr(all\,apples\,good) = \frac{\frac{90!}{80!}}{\frac{100!}{90!}} \approx 0.33$
\end{enumerate}
\bigskip

\section{Multiple events, conditioning, and independence}
\subsection{Worksheet 3}
\begin{enumerate}
\item
	\begin{enumerate}
	\item Looking for $Pr(\{HT\},\{TH\})$. Hence, $Pr(2H|H) = \frac{1}{2}$
	\item Looking for $Pr(\{HH\})$. Hence, $Pr(2H|T) = \frac{1}{4}$
	\item Looking for $Pr(\{T\})$. Hence, $Pr(2H|2H) = \frac{1}{2}$
	\end{enumerate}
\addtocounter{enumi}{3}
\item
	\begin{enumerate}
	\item $Pr(Heart|Red) = \frac{1}{2}$. Even possibility with a Diamond.
	\item $Pr(>10|Heart) = \frac{4}{13}$. $13$ Heart cards, $4$ higher than $10$.
	\item $Pr(Jack|>10) = \frac{1}{4}$. $4$ suits of cards, each with equal likelihood of being $> 10$.
	\end{enumerate}
\addtocounter{enumi}{1}
\item
	\begin{enumerate}
	\item $Pr(>7|4) = Pr(\{4,5,6\}) = \frac{1}{2}$ 
	\addtocounter{enumii}{1}
	\item $Pr(>7|3) = \frac{Pr(>7 \cap >3)}{Pr(>3)}$\\
	$12$ possibilities on the intersection (counted them up), hence: $\frac{\frac{12}{36}}{\frac{1}{2}} = \frac{2}{3}$
	\end{enumerate}
\addtocounter{enumi}{1}
\item 
	\begin{enumerate}
	\item $Pr(D) = (0.05)*(0.25)\, + \, (0.04)*(0.35)\, + \, (0.02)*(0.4) = 0.0345$
	\item $Pr(F_{1}|D) = \frac{Pr(D|F_{1})*Pr(F_{1})}{Pr(D)}$\\
	$= \frac{(0.05)*(0.25)}{.0345} = 0.3623$
	\end{enumerate}
\item $Pr(M|C) = \frac{Pr(C|M)*Pr(M)}{Pr(C)}$,\\
$Pr(C) = (0.5)*(0.05)\,+\,(0.5)*(0.1) = 0.03$, inserting into original equation:\\
$\frac{(0.05)*(0.5)}{0.03} = \frac{5}{6}$
\addtocounter{enumi}{1}
\item $Pr(Trick|6H) = \frac{Pr(6H|Trick)*Pr(Trick)}{Pr(6H)}\\
= \frac{Pr(6H|Trick)*Pr(Trick)}{Pr(6H|Trick)*Pr(Trick)\,+\,Pr(6H|Fair)*Pr(Fair)}\\
= \frac{1*\frac{1}{65}}{(1*\frac{1}{65})+(\frac{1}{64}*\frac{64}{65}}\\
=\frac{\frac{1}{65}}{\frac{2}{65}} = \frac{1}{2}$
\addtocounter{enumi}{1}
\item $Pr(B|S) = \frac{Pr(S|B)*Pr(B)}{Pr(S)}$,\\
$Pr(S) = (0.75)*(0.1)\,+\,(0.25)*(0.6) = 0.225$, inserting into original equation:\\
$\frac{(0.6)*(0.25)}{0.225} = \frac{2}{3}$
\item
	\begin{enumerate}
	\item Event pairs (1) and (2). Work for (2):\\
	$Pr(A\cup D) = Pr(A)*Pr(D)$ for independence:\\
	$Pr(A) = \frac{1}{2}$, $Pr(D) = \frac{2}{8} = \frac{1}{4}$\\
	$Pr(A\cup D) = \frac{1}{2} * \frac{1}{4} + 0$ as the first roll must be a H, and thus the next 2 rolls must also be H.\\
	Hence, $\frac{1}{2} * \frac{1}{4} = \frac{1}{2} * \frac{1}{4} = \frac{1}{8}$. Events A and D are independent.
	\end{enumerate}
\item Event paris (2) and (4). Work for (4):\\
	$Pr(A\cup B) = Pr(B)*Pr(B)$ for independence:\\
	$Pr(A) = \frac{1}{4}$, $Pr(B) = \frac{1}{13}$\\
	$Pr(A\cup B) = \frac{1}{4} * \frac{4}{51} + \frac{1}{52}*\frac{3}{51}$. In other words, the probability of the intersection is equal to the probability of the first card being a Heart and the second one a 10 \textit{plus} the first card being both a Heart and a 10.\\
	Hence, $\frac{1}{4} * \frac{4}{51} + \frac{1}{52}*\frac{3}{51} = \frac{1}{4} * \frac{1}{13} = \frac{1}{52}$. Events A and B are independent.
\end{enumerate}


\end{document}